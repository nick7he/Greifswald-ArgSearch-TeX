\documentclass{beamer}
\usepackage[utf8]{inputenc}
\usepackage{mdframed}

\newmdtheoremenv{theo}{Definition}

\title{argument search}
\begin{document}
	\begin{frame}
		\tableofcontents
	\end{frame}

	\section{Argumentationstheorie}
	\begin{frame}{Argumente}
		%todo: was ist ein Argument?
		%todo: was ist ein gutes und was ein schlechtes Argument?
		%todo: wie ist ein Argument aufgebaut?
		\begin{quote}
			"[An argument is] a collection of truth-bearers (that is, the things that bear truth and falsity, or are true and false) some of which are offered as reasons for one of them, the conclusion."
			\flushright \tiny Matthew McKeon in: Internet Encyclopedia of Philosophy, Eintrag "Argument"
		\end{quote}
		\begin{itemize}[<+->]
			\item Prämissen $\Rightarrow$ Konklusion
			\item was ist ein gutes und was ein schlechtes Argument?
		\end{itemize}
	\end{frame}
	\begin{frame}{Klassifizierung von Argumenten}
		logisch ("Fehlerhaftigkeit"):
		\begin{itemize}[<+->]
			\item Gültigkeit
			\begin{itemize}
				\item aus den Prämissen lässt sich logisch die Konklusion folgern
			\end{itemize}
			\item Schlüssigkeit
			\begin{itemize}
				\item gültig und die Prämissen sind wahr
			\end{itemize}
		\end{itemize}
		subjektiv ("Überzeugungskraft")		\begin{itemize}[<+->]
			\item Konkretheit
			\item Ausrichtung auf den Adressaten
			\item Formulierung - moralisch/emotional/plausibel
		\end{itemize}
	\end{frame}
	\section{Suchmaschinen und retrieval-Modele}
	\section{bestehende Suchmaschinen}
	\section{Probleme bei der Argumentensuche}
	\section{Lösungsvorschlag/Arbeitshypothese}
	\section{Roadmap}
	\begin{frame}
		%todo: was wollen wir noch tun? Suchmaschinen vergleichen/bewerten..
	\end{frame}
\end{document}
