\documentclass{beamer}
\usepackage[utf8]{inputenc}
\usepackage{mdframed}

\newmdtheoremenv{theo}{Definition}

\title{Argument Search}
\subtitle{Einleitung und aktueller Stand}

\begin{document}

	\begin{frame}
		\maketitle
	\end{frame}

	\begin{frame}{Übersicht}
		\tableofcontents
	\end{frame}

	\section{Argumentationstheorie}
	\begin{frame}{Argumente}
		%todo: was ist ein Argument?
		%todo: was ist ein gutes und was ein schlechtes Argument?
		%todo: wie ist ein Argument aufgebaut?
		\begin{quote}
			"[An argument is] a collection of truth-bearers (that is, the things that bear truth and falsity, or are true and false) some of which are offered as reasons for one of them, the conclusion."
			\flushright \tiny Matthew McKeon in: Internet Encyclopedia of Philosophy, Eintrag "Argument"
		\end{quote}
		\begin{itemize}[<+->]
			\item Prämissen $\Rightarrow$ Konklusion
			\item was ist ein gutes und was ein schlechtes Argument?
		\end{itemize}
	\end{frame}
	\begin{frame}{Klassifizierung von Argumenten}
		logisch ("Fehlerhaftigkeit"):
		\begin{itemize}[<+->]
			\item Gültigkeit
			\begin{itemize}
				\item aus den Prämissen lässt sich logisch die Konklusion folgern
			\end{itemize}
			\item Schlüssigkeit
			\begin{itemize}
				\item gültig und die Prämissen sind wahr
			\end{itemize}
		\end{itemize}
		subjektiv ("Überzeugungskraft")		\begin{itemize}[<+->]
			\item Konkretheit
			\item Ausrichtung auf den Adressaten
			\item Formulierung - moralisch/emotional/plausibel
		\end{itemize}
	\end{frame}
	\section{Suchmaschinen und Retrievalmodelle}

	\section{Bestehende Suchmaschinen}
	\begin{frame}{Existierende Argument-Suchmaschinen}
		\begin{itemize}
			\item args (\url{args.me})
			\item ArgumenText (\url{www.argumentsearch.com})
		\end{itemize}
	\end{frame}
	\begin{frame}{args}
		\begin{itemize}
			\item Entwickelt an der Bauhaus-Universität Weimar
			\item Index bestehend aus 291.440 Argumenten
			\begin{itemize}
				\item geschürft von 5 Debattierportalen
				\item bestehende Struktur der Debatten genutzt
			\end{itemize}
			\item Annotation der Argumente mit ``Pro'' und ``Con''
			\item Implementierung basierend auf Apache UIMA und Apache Lucene
		\end{itemize}
		~\\
		\tiny Wachsmuth, Henning, et al. "Building an argument search engine for the web."
		\textit{Proceedings of the 4th Workshop on Argument Mining}. 2017.
	\end{frame}
	\begin{frame}{ArgumenText}
		\begin{itemize}
			\item Entwickelt an der TU Darmstadt
			\item 400 Mil. heterogene englische Textdokumente
			\begin{itemize}
				\item basierend auf CommonCrawl
				\item keine Debattenstruktur / annotierte Argumente
			\end{itemize}
			\item Indexing und Retrieval basierend auf Elasticsearch
			\item Identifikation der Argumente nach Retrieval durch ein neuronales Netz
		\end{itemize}
		~\\
		\tiny Stab, Christian, et al. "ArgumenText: Searching for Arguments in Heterogeneous Sources."
		\textit{Proceedings of the 2018 Conference of the North American Chapter of the Association for
		Computational Linguistics: Demonstrations}. 2018.
	\end{frame}

	\section{Probleme bei der Argumentensuche}
	\section{Lösungsvorschlag/Arbeitshypothese}

	\section{Roadmap}
	\begin{frame}{Roadmap}
		%todo: was wollen wir noch tun? Suchmaschinen vergleichen/bewerten..
		\begin{enumerate}
			\item Implementierung einer Argument-Suchmaschine
			\begin{itemize}
				\item Terrier IR Platform
				\item \url{args.me}-Datensatz
			\end{itemize}
			\item Implementierung eines Feedbacksystems zur Bewertung von Argumenten
			\begin{itemize}
				\item Sophismen / Scheinargumente erkennen und kenntlich machen
				\item Einbeziehung des Feedbacks in das Ranking der Argumente
			\end{itemize}
			\item Evaluierung der Suchmaschine mit Feedback gegen die Suchmaschine ohne Feedback\\(oder evtl. eine 3. Suchmaschine)
			\begin{itemize}
				\item Test anhand ausgewählter Suchanfragen
				\item evtl. Pooling der Suchergebnisse
			\end{itemize}
		\end{enumerate}
	\end{frame}
	\begin{frame}{Aktueller Stand}
		\begin{itemize}
			\item erste Version einer Suchmaschine implementiert
			\begin{itemize}
				\item \url{args.me}-Datensatz für Terrier indiziert
				\item Suche über einheitliche API möglich
			\end{itemize}
			% todo: auf aktuellen Stand bringen
		\end{itemize}
	\end{frame}
\end{document}
