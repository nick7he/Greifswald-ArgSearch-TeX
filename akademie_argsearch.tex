\documentclass{beamer}
\usepackage[ngerman]{babel}
\usepackage[utf8]{inputenc}
\usepackage{mdframed}

\newmdtheoremenv{theo}{Definition}

\title{Argument Search}
\subtitle{Einleitung und aktueller Stand}
\author{Nico Weidmann \and Nick Heilenkötter \and Florian Euchner}
\date{20. August 2018}

\begin{document}
	\begin{frame}
		\maketitle
	\end{frame}

	\begin{frame}{Gliederung}
		\tableofcontents
	\end{frame}

	\section{Argumentationstheorie}
	\begin{frame}{Argumente}
		%todo: was ist ein Argument?
		%todo: was ist ein gutes und was ein schlechtes Argument?
		%todo: wie ist ein Argument aufgebaut?
		\begin{quote}
			"[An argument is] a collection of truth-bearers (that is, the things that bear truth and falsity, or are true and false) some of which are offered as reasons for one of them, the conclusion."
			\flushright \tiny Matthew McKeon in: Internet Encyclopedia of Philosophy, Eintrag "Argument"
		\end{quote}
		\begin{itemize}[<+->]
			\item Prämissen $\Rightarrow$ Konklusion
			\item was ist ein gutes und was ein schlechtes Argument?
		\end{itemize}
	\end{frame}
	\begin{frame}{Klassifizierung von Argumenten}
		logisch ("Fehlerhaftigkeit"):
		\begin{itemize}[<+->]
			\item Gültigkeit
			\begin{itemize}
				\item aus den Prämissen lässt sich logisch die Konklusion folgern
			\end{itemize}
			\item Schlüssigkeit
			\begin{itemize}
				\item gültig und die Prämissen sind wahr
			\end{itemize}
		\end{itemize}
		subjektiv ("Überzeugungskraft")		\begin{itemize}[<+->]
			\item Konkretheit
			\item Ausrichtung auf den Adressaten
			\item Formulierung - moralisch/emotional/plausibel
		\end{itemize}
	\end{frame}

	\section{Suchmaschinen und Retrievalmodelle}

	\section{Bestehende Suchmaschinen}
	\begin{frame}{Existierende Argument-Suchmaschinen}
		\begin{itemize}
			\item args (\url{args.me})
			\item ArgumenText (\url{www.argumentsearch.com})
		\end{itemize}
	\end{frame}
	\begin{frame}{args}
		\begin{itemize}
			\item Entwickelt an der Bauhaus-Universität Weimar
			\item Index bestehend aus 291.440 Argumenten
			\begin{itemize}
				\item geschürft von 5 Debattierportalen
				\item bestehende Struktur der Debatten genutzt
			\end{itemize}
			\item Annotation der Argumente mit ``Pro'' und ``Con''
			\item Implementierung basierend auf Apache UIMA und Apache Lucene
		\end{itemize}
		~\\
		\tiny Wachsmuth, Henning, et al. "Building an argument search engine for the web."
		\textit{Proceedings of the 4th Workshop on Argument Mining}. 2017.
	\end{frame}
	\begin{frame}{ArgumenText}
		\begin{itemize}
			\item Entwickelt an der TU Darmstadt
			\item 400 Mil. heterogene englische Textdokumente
			\begin{itemize}
				\item basierend auf CommonCrawl
				\item keine Debattenstruktur / annotierte Argumente
			\end{itemize}
			\item Indexing und Retrieval basierend auf Elasticsearch
			\item Identifikation der Argumente nach Retrieval durch ein neuronales Netz
		\end{itemize}
		~\\
		\tiny Stab, Christian, et al. "ArgumenText: Searching for Arguments in Heterogeneous Sources."
		\textit{Proceedings of the 2018 Conference of the North American Chapter of the Association for
		Computational Linguistics: Demonstrations}. 2018.
	\end{frame}

	\section{Probleme bei der Argumentensuche}
	\begin{frame}{Probleme bei der Argumentensuche}{Warum nicht Google?}
		\begin{itemize}[<+->]
			\item Suchbegriff nimmt Antwort vorweg: Keine neutrale, umfassende Information
			\alt <1, 2>{
				\begin{figure}
					\includegraphics<1>[width=0.8\textwidth]{img/s21_pro.png}
					\includegraphics<2>[width=0.8\textwidth]{img/s21_contra.png}
					\caption{Google-Suche nach S21 mit positiv / negativ konnotierten Begriffen}
				\end{figure}
			}{}
			\item<3-> Ergebnisse nach Relevanz, nicht Qualität sortiert
			\item<4-> Keine Trennung von Thesen und Argumenten
			\item<5-> Schwer verständlich: Lange Artikel, viel Text
			\item<6-> Erkennung von Trugschlüssen, Verschwörungstheorien bleibt Nutzer überlassen
			\item<7-> Suchmaschinen nehmen gesellschaftliche Verantwortung kaum wahr
		\end{itemize}

		
	\end{frame}

	\begin{frame}{Probleme bei der Argumentensuche}{Schlechte Argumente - Trugschlüsse (\textit{fallacies})}
		\begin{itemize}[<+->]
			\item Persönlicher Angriff (\textit{ad hominem})
			\begin{quote}
				\small You are the single biggest liar, you probably are worse than Jeb Bush.
				\begin{flushright}
					\tiny Donald Trump zu Ted Cruz
				\end{flushright}
			\end{quote}
			\item Strohmann (\textit{straw man})
			\begin{quote}{}
				\small Over and over, we have been told by our opponents that bigger tax cuts and fewer regulations are the only way.
				\begin{flushright}
					\tiny Barack Obama vor der DNC, 2012
				\end{flushright}
			\end{quote}
			\item Falsches Dilemma (\textit{false dichotomy})
			\begin{quote}{}
				\small Every nation, in every region, now has a decision to make. Either you are with us, or you are with the terrorists.
				\begin{flushright}
					\tiny George W. Bush nach 9/11
				\end{flushright}
			\end{quote}
			\item uvm. \footnote{Siehe auch \href{https://en.wikipedia.org/wiki/List_of_fallacies}{Wikipedia: List of fallacies}}
		\end{itemize}
	\end{frame}

	\begin{frame}{Probleme bei der Argumentensuche}{Psychologie - Kognitive Verzerrungen (\textit{cognitive biases})}
		\begin{itemize}
			\item Bestätigungsfehler (\textit{confirmation bias})
			\item Congruence bias
			\item Dunning–Kruger effect
			\item Stereotypisierung
			\item Gerechte-Welt-Glaube (\textit{just-world hypothesis})
			\item uvm. \footnote{Siehe auch \href{https://en.wikipedia.org/wiki/List_of_cognitive_biases}{Wikipedia: List of cognitive biases}}
		\end{itemize}
	\end{frame}

	\section{Lösungsvorschlag / Arbeitshypothese}
	\begin{frame}{Lösungsvorschlag / Arbeitshypothese}{Die ideale Suchmaschine}
		\begin{block}{Informationsbeschaffung mit Argumentensuche sollte sein...}
			\begin{itemize}
				\item Zum Suchbegriff passend
				\item Neutral, verschiedene Thesen mit ``Pro'' und ``Contra''
				\item Umfassend, Argumente nach Qualität sortiert
				\item Leicht und schnell verständlich
				\item Trugschlüsse und ``kontroverse Theorien'' werden nicht versteckt, aber als solche markiert
				\item Menschliche Psychologie sollte beim Design einer Suchmaschine berücksichtigt werden
			\end{itemize}
		\end{block}
		\begin{block}{Grundproblem: Filter bubbles / Echokammern / Fake news}
			Suchmaschine soll auch bei kontroversen Themen funktionieren!
		\end{block}
	\end{frame}

	\begin{frame}{Arbeitshypothese}
		Die Qualität der Suchergebnisse (Argumente) kann verbessert werden durch
		\begin{itemize}[<+->]
			\item \textbf{Nutzerbewertungen}: Nutzer können Argumente als gut befinden und \textit{fallacies} markieren
			\item Trennung von These und Thema: Argumente sind immer an eine \textbf{These} (\textit{plastic bags should be banned}) und nicht an ein \textbf{Thema} (\textit{plastic bags}) geknüpft
		\end{itemize}
	\end{frame}

	\section{Roadmap}
	\begin{frame}{Roadmap}
		%todo: was wollen wir noch tun? Suchmaschinen vergleichen/bewerten..
		\begin{enumerate}
			\item Implementierung einer Argument-Suchmaschine
			\begin{itemize}
				\item Terrier IR Platform
				\item \url{args.me}-Datensatz
			\end{itemize}
			\item Implementierung eines Feedbacksystems zur Bewertung von Argumenten
			\begin{itemize}
				\item Sophismen / Scheinargumente erkennen und kenntlich machen
				\item Einbeziehung des Feedbacks in das Ranking der Argumente
			\end{itemize}
			\item Evaluierung der Suchmaschine mit Feedback gegen die Suchmaschine ohne Feedback\\(oder evtl. eine 3. Suchmaschine)
			\begin{itemize}
				\item Test anhand ausgewählter Suchanfragen
				\item evtl. Pooling der Suchergebnisse
			\end{itemize}
		\end{enumerate}
	\end{frame}
	\begin{frame}{Aktueller Stand}
		\begin{itemize}
			\item erste Version einer Suchmaschine implementiert
			\begin{itemize}
				\item \url{args.me}-Datensatz für Terrier indiziert
				\item Suche über einheitliche API möglich
			\end{itemize}
			% todo: auf aktuellen Stand bringen
		\end{itemize}
	\end{frame}
\end{document}
